\chapter{System Execution Modeling \\
  \small{\textit{-- Evan Ciok, Sophia DiCuffa, Carson McManus}}
  \index{Chapter!labthree}
  \index{Lab Three}
  \label{Chapter::LabThree}}


\section{System Execution Modeling \label{Section::LabThree}}

\subsection{Question 1}
Bottleneck = Merchant’s Server CPU, because it has the highest service demand.
Maximum throughput = (1 / sum of all service demands) = $1/.548$ = $1.82$ tps.

\subsection{Question 2}
The CPU demands for the merchant server and payment gateway will decrease over the next five years.
\begin{center}
  \begin{tabular}{ |c|c|c| }
    \hline
    Year & Merchant Server CPU (ms) & Payment Gateway CPU(ms) \\
    \hline
    1    & 245.9                    & 102.5                   \\
    \hline
    2    & 153.69                   & 64.06                   \\
    \hline
    3    & 95.70                    & 40.04                   \\
    \hline
    4    & 61.32                    & 25.56                   \\
    \hline
    5    & 38.36                    & 15.99                   \\
    \hline
  \end{tabular}
\end{center}

\subsection{Question 3}
Service demands of the new algorithm are 37 \% of the service demands in year one.
Updated merchant server CPU demand: 0.37 * 245.9 = 90.983
Updated payment gateway CPU demand: 0.37 * 102.5 = 37.925
\begin{center}
  \begin{tabular}{ |c|c|c| }
    \hline
    Year & Merchant Server CPU (ms) & Payment Gateway CPU(ms) \\
    \hline
    3    & 35.54                    & 14.81                   \\
    \hline
    4    & 22.69                    & 9.46                    \\
    \hline
    5    & 14.19                    & 5.92                    \\
    \hline
  \end{tabular}
\end{center}

\subsection{Question 4}

Total service demand (ms) = $190 + 1.6 + 245.9 + 8 + 102.5 = 548$

This means that 1 CPU can handle $1/(548/1000) = 1.824817518$ transations/sec

We can use this relationship to calculate the number of CPUs required to handle the desired transaction rate.

First, we need to calculate how the service demand changes over time.

\begin{center}
  Merchant CPU service demands over time
  \begin{tabular}{ |c|c|c|c| }
    \hline
    Year & Processing Speedup & Without algorithm (ms) & With algorithm (ms) \\
    \hline
    1    & 1                  & 245.90                 & 90.98               \\
    \hline
    2    & 1.6                & 153.69                 & 56.86               \\
    \hline
    3    & 2.56               & 96.05                  & 35.54               \\
    \hline
    4    & 4.01               & 61.32                  & 22.69               \\
    \hline
    5    & 6.41               & 38.36                  & 14.19               \\
    \hline
  \end{tabular}
\end{center}

\begin{center}
  Payment Gateway CPU service demands over time
  \begin{tabular}{ |c|c|c|c| }
    \hline
    Year & Processing Speedup & Without algorithm (ms) & With algorithm (ms) \\
    \hline
    1    & 1                  & 102.50                 & 37.93               \\
    \hline
    2    & 1.6                & 64.06                  & 23.70               \\
    \hline
    3    & 2.56               & 40.04                  & 14.81               \\
    \hline
    4    & 4.01               & 25.56                  & 9.46                \\
    \hline
    5    & 6.41               & 15.99                  & 5.92                \\
    \hline
  \end{tabular}
\end{center}

Now we can use these values to calculate the number of CPUs required to handle the desired transaction rate, taking into account that the algorithm comes into effect in year 3. We need to round up the number of CPUs required to the nearest integer because we can't have a fraction of a CPU.

\begin{equation}
  CPUs = ceil(\frac{\text{target tps}}{\text{percent of SET transactions} / \text{service demand in seconds}})
\end{equation}

To get the number of CPUs required for them to only be loaded at 60\% instead of 100\%:

\begin{equation}
  CPUs = ceil(\text{CPUs at 100\%} * 1.4)
\end{equation}

\begin{center}
  Merchant CPU (without algorithm)
  \begin{tabular}{ |c|c|c|c|c| }
    \hline
    Year & Target tps & 1 CPU SET tps & CPUs required & CPUs for 60\% \\
    \hline
    1    & 14         & 1.626677511   & 9             & 13            \\
    \hline
    2    & 19         & 2.602684018   & 8             & 12            \\
    \hline
    3    & 23         & 4.164294429   & 6             & 9             \\
    \hline
    4    & 26         & 6.52297682    & 4             & 6             \\
    \hline
    5    & 29         & 10.42700285   & 3             & 5             \\
    \hline
  \end{tabular}
\end{center}

\begin{center}
  Merchant CPU (with algorithm)
  \begin{tabular}{ |c|c|c|c|c| }
    \hline
    Year & Target tps & 1 CPU SET tps & CPUs required & CPUs for 60\% \\
    \hline
    1    & 14         & 1.626677511   & 9             & 13            \\
    \hline
    2    & 19         & 2.602684018   & 8             & 12            \\
    \hline
    3    & 23         & 11.25484981   & 3             & 5             \\
    \hline
    4    & 26         & 17.62966708   & 2             & 3             \\
    \hline
    5    & 29         & 28.18108877   & 2             & 3             \\
    \hline
  \end{tabular}
\end{center}

\begin{center}
  Payment Gateway CPU (without algorithm)
  \begin{tabular}{ |c|c|c|c|c| }
    \hline
    Year & Target tps & 1 CPU SET tps & CPUs required & CPUs for 60\% \\
    \hline
    1    & 14         & 9.76          & 2             & 3             \\
    \hline
    2    & 19         & 15.61         & 2             & 3             \\
    \hline
    3    & 23         & 24.98         & 1             & 2             \\
    \hline
    4    & 26         & 39.12         & 1             & 2             \\
    \hline
    5    & 29         & 62.54         & 1             & 2             \\
    \hline
  \end{tabular}
\end{center}

\begin{center}
  Payment Gateway CPU (without algorithm)
  \begin{tabular}{ |c|c|c|c|c| }
    \hline
    Year & Target tps & 1 CPU SET tps & CPUs required & CPUs for 60\% \\
    \hline
    1    & 14         & 9.76          & 2             & 3             \\
    \hline
    2    & 19         & 15.61         & 2             & 3             \\
    \hline
    3    & 23         & 67.50         & 1             & 2             \\
    \hline
    4    & 26         & 105.74        & 1             & 2             \\
    \hline
    5    & 29         & 169.02        & 1             & 2             \\
    \hline
  \end{tabular}
\end{center}


